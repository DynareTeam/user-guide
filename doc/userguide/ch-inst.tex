\chapter{Installing Dynare} \label{ch:inst}

\section{System requirements}
Dynare is supported by Windows 98, 2000, NT and XP; at the time of writing, no experience was available on Vista (** change this). Dynare can also run on Unix (please write to Michel Juillard with questions about support for particular platforms ** is this still the case?) and Mac OS X. \\

To run Dynare, it is recommended to allocate at least 256MB of RAM to the platform running Dynare, although 512MB is preferred. Depending on the type of computations required, like the very processor intensive Metropolis Hastings algorithm, you may need up to 1GB of RAM to obtain acceptable computational times. \\

\section{Installation}
Three versions of Dynare exist: one for Matlab, one for Scilab and one for Gauss. The first benefits from ongoing development and is the most popular. Development of the Scilab version stopped after Dynare version 3.02 and that for Gauss after Dynare version 1.2. \\

This User Guide will focus exclusively on the Matlab version of Dynare. For the installation procedure for the Scilab or Gauss versions of the program, please see the \href{http://www.cepremap.cnrs.fr/juillard/mambo/index.php?option=com_content&task=view&id=51&Itemid=84}{Reference Manual}. Note, though, that the main functionality - and especially syntax - of Dynare remains mostly unchanged across the Matlab, Scilab or Gauss versions, for those features common to the older versions of Dynare. \\

\subsection{Installing Dynare for Matlab on Windows}
The following assumes you have Matlab version 6.5.1 or later installed on your Windows system.
\begin{enumerate}
\item Download the latest stable version of Dynare for Matlab (Windows) from the \href{http://www.cepremap.cnrs.fr/juillard/mambo/index.php?option=com_frontpage&Itemid=1}{Dynare website}. 
\item You will now have on your computer a .zip file which you should unzip. This will create a folder called, by default, Dynare and its version number, for example: Dynare\_v3.0 
\item This directory contains several sub-directories, among which (i) matlab, (ii) doc and (iii) examples. 
\item Place the Dynare folder (Dynare\_v3.0 in our example) in the c: directory and note that location. The easiest is probably to put it in the root of c: as in c:/dynare\_v3.0.
\item Start Matlab and use the menu File/Set-Path to add the path to the Dynare 
matlab subdirectory. Following our example, this would correspond to 
c:/dynare\_v3.0/matlab
\item Save these changes in Matlab and you're ready to go. (** doesn't work if dynare is put in program files, for instance... strange?)
\end{enumerate} 

\subsection{Installing Dynare for Matlab on UNIX}
** TBD - must recompile parser, but need exact instructions. 
\subsection{Installing Dynare for Matlab on Mac OSX}
** TBD - must recompile parser, but need exact instructions. 

\section{Matlab particularities}

A question often comes up: what special Matlab toolboxes are necessary to run Dynare? In fact, no additional toolbox is necessary for running most of Dynare, except maybe for optimal simple rules (see chapter \ref{ch:ramsey}), but even then remedies exist (see the \href{http://www.cepremap.cnrs.fr/juillard/mambo/index.php?option=com_forum&Itemid=95}{Dynare forums} for discussions on this, or ask your particular question there). But if you do have the 'optimization toolbox' installed, you will have additional options for solving for the steady state (solve\_algo option) and for searching for the posterior mode (mode\_compute option), defined later. 


